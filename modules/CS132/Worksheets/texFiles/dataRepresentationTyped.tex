\documentclass{article}
\usepackage[utf8]{inputenc}
\usepackage{tikz}
\usepackage{graphicx}
\graphicspath{ {./images/} }


\title{CS132 Quizzes - Data Representation}
\begin{document}
\begin{center}
    \Huge\textbf{CS132 Quizzes - Data Representation}\\
    \huge\textit{May 2021}\\
    \medskip
    \Large\textit{Josh Fitzmaurice}
\end{center}




\section{Briefly describe why binary code is commonly used in computer hardware}
Computer use electricity to send signals around the components. These signals
can be of varying voltages. There are levels of voltage that determine whether
the signal is high or low. We only have a high and a low because noise can make
the voltage amount vary slightly. Having binary code where the voltage is either
high or low limits the affect of noise in a system.

\section{How many bits in a byte}
8

\section{In the binary number $10101010_2$ what is the value of the MSB?}
1
\newpage
\section{Make a table counting upwards from 0 to $16_{10}$ in decimal, binary,
octal and hexadecimal.}
\begin{table}[h]
    \centering
    \begin{tabular}{|c|c|c|c|}
        Binary & Octal & Decimal & Hexadecimal \\
        0 & 0 & 0 & 0 \\
        1 & 1 & 1 & 1 \\
        10 & 2 & 2 & 2 \\
        11 & 3 & 3 & 3 \\
        100 & 4 & 4 & 4 \\
        101 & 5 & 5 & 5 \\
        110 & 6 & 6 & 6 \\
        111 & 7 & 7 & 7 \\
        1000 & 10 & 8 & 8 \\
        1001 & 11 & 9 & 9 \\
        1010 & 12 & 10 & A \\
        1011 & 13 & 11 & B \\
        1100 & 14 & 12 & C \\
        1101 & 15 & 13 & D \\
        1110 & 16 & 14 & E \\
        1111 & 17 & 15 & F \\
        10000 & 20 & 16 & 10
    \end{tabular}
    \caption{numbers from 1 - 16}
    \label{tab:my_label}
\end{table}

\section{Briefly explain the difference between value and representation, giving an example.}
Representation is how we show values and can change with different
representations. Whereas the value is set and even though you can represent a
value in different ways the value will remain constant. E.g. 13 in decimal is
1101 in binary or 15 in Octal.

\section{Which of the following are not valid hex values?}
a - valid\\
b - valid\\
c - invalid\\
d - valid\\
e - invalid

\section{What is $2742_8$ in binary?}
010 111 100 010

\section{Convert $1011 0010 1111 1001_2$ to hex}
B2F9

\section{Convert $42_{10}$ to binary} 101010

\section{Convert $73_8$ to hex.}
111

\section{Convert $1101100100_2$ to decimal.}
868

\section{Convert $4000_{10}$ to octal.} $111 110 100 000_2$\\
$7640_8$

\section{Calculate the following binary sum: 10100111+01110001}
100011000\\
assuming we are allowing an overflow

\section{Calculate the following binary sum: 10111+11011}
110010\\
assuming we are allowing an overflow

\section{Show the binary representations for $–13_{10}$ in\\
a. signed magnitude and\\
b. two’s complement. } a. 11101 b. 13 = 01101 Flip the bits\\
10010\\
add 1\\
10011

\section{a. Find the binary two’s complement representations of $+12_{10}$ and
$–10_{10}$.\\
b. Use your answers to subtract 10 from 12. Show your working} 12 = 01100\\
10 = 01010\\
flip bits\\
   = 10101\\
add 1\\
-10 = 10110\\
\\
$12-10 = 12 + (-10)  = 01100 + 10110 = 00010$\\
Remember we remove the overflow.

\section{Do the following statements describe fixed or floating point
representations, both or neither?\\
a. It’s fast\\
b. Provides the best resolution\\
c. Copes with a wide range of numbers\\
d. Implementation is complicated\\
e. Can’t represent some values\\
f. Is described by an international standard\\
g. Can represent any value\\
h. Allows simple multiplication by two} a - fixed b - fixed c - floating d -
floating e - both f - floating g - neither h - both

\section{Using 4 bit binary arithmetic, illustrate overflow error with an example. }
1101 + 0100 = 10001\\
the MSB is an overflow error in this example.
\newpage
\section{Describe IEEE 754 single precision floating point representation using a labelled diagram.}
The MSB represents the sign of the number 1 for negative 0 for positive (we'll
denote this as s).\\
The next 8 MSB's are the exponent (we'll denote as e)\\
The final 23 bits are the fraction (we'll denote as f)\\
We then calculate the value using the following formula:\\
$(-1)^s \times 1.f \times 2^{e - 127}$\\
\begin{table}[h]
    \centering
    \begin{tabular}{|c|c|c|}
        Sign bit & Exponent & Fraction \\
        1 & 10001010 & 1101000000000000000000\\
        \hline
        1 & 11 & 0.8125
    \end{tabular}
    \caption{example}
    \label{tab:my_label}
\end{table}

$(-1)^1 \times 1.8125 \times 2^{11} = -3712$\\
There are also some special values.


\end{document}
